%%%%%%%%%%%%%%%%%%%%%%%%%%%%%%%%%%%%%%%%%%%%%%%%%%%%%%%%%%%%%%%%%%%%%%
%%  Copyright by Wenliang Du.                                       %%
%%  This work is licensed under the Creative Commons                %%
%%  Attribution-NonCommercial-ShareAlike 4.0 International License. %%
%%  To view a copy of this license, visit                           %%
%%  http://creativecommons.org/licenses/by-nc-sa/4.0/.              %%
%%%%%%%%%%%%%%%%%%%%%%%%%%%%%%%%%%%%%%%%%%%%%%%%%%%%%%%%%%%%%%%%%%%%%%

\input{../../common-files/header}
\input{../../common-files/copyright}


\lhead{\bfseries SEED Labs -- Laboratório de Ataque de Envenenamento do Cache ARP}
\newcommand{\arpFigs}{./Figs}


\begin{document}
	
	
	
	\begin{center}
		{\LARGE Laboratório de Ataque de Envenenamento do Cache ARP}
	\end{center}
	
	\seedlabcopyright{2019}
	
	
	
	% *******************************************
	% SECTION
	% ******************************************* 
	\section{Visão geral}
	
	
	O protocolo de resolução de endereços (ARP) é um protocolo de comunicação usado para descobrir o link
	endereço de camada, como o endereço MAC, dado um endereço IP. O protocolo ARP é muito simples
	protocolo e não implementa nenhuma medida de segurança.
	O ataque de envenenamento de cache ARP é um ataque comum contra o protocolo ARP.
	Usando esse tipo de ataque, os invasores podem fazer a vítima aceitar
	mapeamentos IP para MAC forjados. Isso pode fazer com que os pacotes da vítima sejam
	redirecionado para o computador com o endereço MAC forjado, levando a
	potenciais ataques man-in-the-middle.
	
	
	O objetivo deste laboratório é que os alunos obtenham a experiência prática no ataque de envenenamento da tabela ARP e aprenda quais danos podem ser causados ​​por tal ataque.
	Em particular, os alunos usarão o ataque ARP para lançar um ataque man-in-the-middle,
	onde o invasor pode interceptar e modificar os pacotes entre as duas vítimas A e B.
	Outro objetivo deste laboratório é que os alunos pratiquem
	habilidades de sniffing e spoofing de pacotes, pois estas são
	habilidades essenciais em segurança de rede, e são os blocos de construção
	para muitas ferramentas de ataque e defesa de rede.
	Os alunos usarão o Scapy para realizar tarefas de laboratório.
	Este laboratório cobre os seguintes tópicos:
	
	\begin{itemize}[noitemsep]
		\item O protocolo ARP
		\item O ataque de envenenamento do cache ARP
		\item Ataque Man-in-the-middle
		\item Programando o Scapy
	\end{itemize}
	
	
	
	\paragraph{Videos.}
	Detalhes sobre o protocolo ARP protocol e respectivos ataques podem ser encontrados em:
	
	\begin{itemize}
		\item Seção 3 do curso SEED Lecture na Udemy, \seedisvideo
	\end{itemize}
	
	
	\paragraph{Ambiente do Laboratório.} \seedenvironment
	
	
	% *******************************************
	% SECTION
	% ******************************************* 
	\section{Tarefa 1: Envenenamento do cache ARP}
	
	
	O objetivo desta tarefa é usar o spoofing de pacotes para lançar um ataque de envenenamento de cache ARP em
	um alvo, de modo que, quando duas máquinas vítimas A e B tentam se comunicar, seus 
	pacotes serão interceptados pelo invasor, que pode fazer alterações nos pacotes e, portanto,
	tornar-se o "homem no meio" (man-in-the-middle) entre A e B. Isso é chamado de ataque Man-In-The-Middle (MITM).
	Neste laboratório, usamos envenenamento de cache ARP para conduzir um ataque MITM. 
	
	
	
	O código abaixo mostra como criar um pacote ARP usando o Scapy.
	
	\begin{lstlisting}
	#!/usr/bin/python3
	from scapy.all import *
	
	E = Ether()
	A = ARP()
	
	pkt = E/A
	sendp(pkt)
	\end{lstlisting}
	
	O programa acima constrói e envia um pacote ARP. É necessário definir os nomes/valores de atributos 
	necessários para definir seu próprio pacote ARP. Podemos usar \texttt{ls (ARP)} para ver o atributo
	de nomes da classe ARP. Se um campo não for definido, um valor padrão será usado (veja a terceira
	coluna da saída):
	
	\begin{lstlisting}
	$ python3
	>>> from scapy.all import *
	>>> ls(ARP)
	hwtype     : XShortField                         = (1)
	ptype      : XShortEnumField                     = (2048)
	hwlen      : ByteField                           = (6)
	plen       : ByteField                           = (4)
	op         : ShortEnumField                      = (1)
	hwsrc      : ARPSourceMACField                   = (None)
	psrc       : SourceIPField                       = (None)
	hwdst      : MACField                            = ('00:00:00:00:00:00')
	pdst       : IPField                             = ('0.0.0.0')
	\end{lstlisting}
	
	Nesta tarefa, temos três VMs, A, B e M. Gostaríamos de atacar o cache ARP de A, de forma que
	os seguintes resultados sejam obtidos no cache ARP de A.
	
	\begin{lstlisting}
	B's IP address --> M's MAC address
	\end{lstlisting}
	
	Existem muitas maneiras de conduzir um ataque de envenenamento do cache ARP. Os alunos precisam 
	tentar os três métodos a seguir e relatar se cada método funciona ou não.
	
	
	\begin{itemize}
		\item \textbf{Tarefa 1A (usando requisição ARP).} No Host M, construa um pacote de solicitação ARP e envie
		para o Host A. Verifique se o endereço MAC de M está mapeado para o endereço de IP do Host B no cache ARP do Host A.
		
		
		\item \textbf{Tarefa 1B (usando resposta ARP).} On Host M, construct an ARP reply packet and send to
		Host A. Check whether M's MAC address is mapped to B's IP address in A's ARP cache.
		
		No Host M, construa um pacote de resposta ARP e envie para o
		Host A. Verifique se o endereço MAC de M está mapeado para o endereço de IP do Host B no cache ARP do Host A.
		
		
		
		\item \textbf{Tarefa 1C (usando mensagem ARP gratuita).} No Host M, construa um pacote ARP gratuito. O pacote ARP gratuito 
		é um pacote especial de solicitação ARP. É usado quando um Host precisa atualizar 
		informações desatualizadas em todo o cache ARP da outra máquina. O pacote de ARP gratuito
		tem as seguintes características:
		
		
		\begin{itemize}
			\item Os endereços IP de origem e destino são os mesmos e são 
			o endereço IP do Host que está emitindo o ARP gratuito.
			
			\item Os endereços MAC de destino no cabeçalho ARP e no cabeçalho Ethernet são o endereço 
			MAC de transmissão ({\tt ff:ff:ff:ff:ff:ff}).
			
			\item Nenhuma resposta é esperada.
		\end{itemize}
	\end{itemize}
	
	
	
	% *******************************************
	% SECTION
	% ******************************************* 
	\section{Tarefa 2: Ataque MITM no Telnet usando envenenamento do Cache ARP Cache}
	
	Os Hosts A e B estão se comunicando usando Telnet e o Host M deseja interceptar essa
	comunicação para que possa fazer alterações nos dados enviados entre A e B. A configuração é representada
	na figura~\ref{arp:fig:telnet_mitm}. 
	
	\begin{figure}
		\centering
		\includegraphics[width=0.8\textwidth]{\arpFigs/telnet_mitm.pdf}
		\caption{Man-In-The-Middle Attack against telnet}
		\label{arp:fig:telnet_mitm}
	\end{figure}
	
	
	\paragraph{Passo 1 (Inicie o ataque de envenenamento de cache ARP).} Primeiro, o Host M conduz um 
	ataque de envenenamento docache ARP nos Hosts A e B, de modo que no cache ARP de A, o endereço IP de B mapeie para o MAC de M
	e no cache ARP de B, o endereço IP de A também mapeie para o endereço MAC de M. Após essa etapa,
	pacotes enviados entre A e B serão todos enviados para M. Usaremos o ataque de envenenamento de cache ARP
	da Tarefa 1 para atingir este objetivo. 
	
	
	\paragraph{Passo 2 (Testando).} Depois que o ataque for bem-sucedido, tente fazer ping um no outro
	entre os Hosts A e B, e relate sua observação. Por favor, mostre os resultados do Wireshark em seu
	relatório.
	
	\paragraph{Passo 3 (Ative o encaminhamento de IP).} Agora, ativado o encaminhamento de IP no Host M, para que ele possa
	encaminhar os pacotes entre A e B. Execute o seguinte comando e repita o Passo 2.
	Descreva a sua observação.
	
	\begin{lstlisting}
	$ sudo sysctl net.ipv4.ip_forward=1
	\end{lstlisting}
	
	\paragraph{Passo 4 (Execute o ataque MITM).} Estamos prontos para fazer alterações nos dados Telnet
	entre A e B. Suponha que A seja o cliente Telnet e B o servidor Telnet. Depois que A se conectou 
	ao servidor Telnet em B, para cada tecla digitada na janela Telnet de A, um TCP
	pacote é gerado e enviado para B. Gostaríamos de interceptar o pacote TCP e substituir cada 
	caractere digitado com um caractere fixo (digamos Z). Dessa forma, não importa o que o usuário
	digitar em A, o Telnet sempre exibirá Z.
	
	Das etapas anteriores, somos capazes de redirecionar os pacotes TCP para o Host M, mas em vez de
	encaminhá-los, gostaríamos de substituí-los por um pacote falsificado. Vamos escrever um
	programa sniff-and-spoof para atingir esse objetivo. Em particular, gostaríamos de fazer o
	seguinte:
	
	\begin{itemize}
		
		\item Primeiro, mantemos o encaminhamento de IP ativado, para que possamos criar uma conexão Telnet com sucesso
		entre A e B. Assim que a conexão for estabelecida, desligamos o encaminhamento de IP usando o
		seguinte comando. Digite algo na janela Telnet de A e relate sua observação: 
		
		\begin{lstlisting}
		$ sudo sysctl net.ipv4.ip_forward=0
		\end{lstlisting}
		
		\item Executamos nosso programa sniff-and-spoof no Host M, de modo que para os pacotes capturados enviados
		de A para B, falsificamos um pacote, mas com dados TCP diferentes. Para pacotes de B para A (resposta Telnet), 
		não fazemos nenhuma alteração, então o pacote falsificado é exatamente o mesmo que o original.
	\end{itemize} 
	
	Para ajudar os alunos a começar, oferecemos um exemplo de um programa sniff-and-spoof. 
	O programa captura todos os pacotes TCP, e
	então para pacotes enviados de A para B, ele faz algumas mudanças (a parte da modificação
	não está incluída, porque faz parte da tarefa). Para pacotes enviados de
	B para A, o programa simplesmente encaminha os pacotes originais.
	
	\begin{lstlisting}
	#!/usr/bin/python3
	from scapy.all import *
	
	VM_A_IP = "10.0.2.6"
	VM_B_IP = "10.0.2.7"
	
	def spoof_pkt(pkt):
	if pkt[IP].sr == VM_A_IP and pkt[IP].dst == VM_B_IP \
	and pkt[TCP].payload:
	
	# Cria um novo pacote com base no pacote capturado.
	# (1) Precisamos excluir os campos de soma de verificação nos cabeçalhos IP e TCP,
	# porque nossa modificação os tornará inválidos.
	# O Scapy irá recalculá-los para nós se esses campos estiverem faltando.
	# (2) Excluímos também a carga útil do TCP original.
	newpkt = IP(pkt[IP])
	del(newpkt.chksum)  
	del(newpkt[TCP].chksum) 
	del(newpkt[TCP].payload) 
	
	#####################################################################
	# Construa a nova carga com base na antiga.
	# Os alunos precisam implementar essa parte.
	
	olddata = pkt[TCP].payload.load   # Get the original payload data 
	newdata = olddata  # No change is made in this sample code
	#####################################################################
	
	# Anexe os novos dados e configure o pacote
	send(newpkt/newdata)
	
	elif pkt[IP].src == VM_B_IP and pkt[IP].dst == VM_A_IP:
	send(pkt[IP]) # Forward the original packet
	
	pkt = sniff(filter='tcp',prn=spoof_pkt)
	\end{lstlisting}
	
	
	Deve-se notar que o código acima captura todo os pacotes TCP,
	incluindo aquele gerado pelo próprio programa. Isso é
	indesejável, pois afetará
	o desempenho. Os alunos precisam mudar o filtro, para que o programa não capture
	seus próprios pacotes.
	
	
	\paragraph{Behavior of Telnet.}
	No Telnet, normalmente, cada caractere que digitamos na seu terminal é acionado
	um pacote TCP individual, mas se você digitar muito rápido, alguns caracteres podem ser
	enviados juntos no mesmo pacote.
	É por isso que em um típico pacote Telnet enviado do cliente para o servidor,
	a carga útil contém apenas um caractere. O 
	caractere enviado ao servidor será ecoado de volta pelo servidor,
	e o cliente então exibirá o
	caractere em sua janela. Portanto, o que vemos na janela do cliente não é o resultado direto
	da digitação; tudo o que digitarmos na janela do cliente fará uma viagem completa antes de ser exibido.
	Se a rede estiver desconectada, tudo o que digitamos na janela do cliente não será exibido,
	até que a rede seja recuperada. Da mesma forma, se os atacantes mudarem o caractere para Z durante o
	ida e volta, Z será exibido na janela do cliente Telnet, embora
	não foi isso que você digitou.
	
	
	
	
	
	% *******************************************
	% SECTION
	% ******************************************* 
	\section{Tarefa 3: Ataque MITM no Netcat usando Envenenamento do cache ARP}
	
	Esta tarefa é semelhante à Tarefa 2, exceto que
	os hosts A e B estão se comunicando usando \texttt{netcat}, ao invés de \texttt{telnet}.
	O Host M deseja interceptar sua
	comunicação, para que possa fazer alterações nos dados enviados entre A e B.
	Você pode usar os seguintes comandos para estabelecer um \ texttt {netcat} TCP
	conexão entre A e B:
	
	
	\begin{lstlisting}
	No Host B (servidor, endereço de IP 10.0.2.7), rode o seguinte comando:
	$ nc -l 9090
	
	No Host A (cliente), rode o seguinte comando:
	$ nc 10.0.2.7 9090
	\end{lstlisting}
	
	
	Assim que a conexão for estabelecida, você pode digitar mensagens em A.
	Cada linha de mensagem será colocada em um pacote TCP enviado
	para B, que simplesmente exibe a mensagem.
	Sua tarefa é substituir todas as ocorrências de seu primeiro nome na
	mensagem com uma sequência de A's. O comprimento da sequência deve ser o
	igual ao do seu primeiro nome, ou você bagunçará a sequência TCP
	número e, portanto, toda a conexão TCP. Você precisa usar seu verdadeiro
	primeiro nome, para sabermos que o trabalho foi feito por você.
	
	
	
	% *******************************************
	% SECTION
	% ******************************************* 
	\section{Submissão}
	
	\seedsubmission
	
	
\end{document}



